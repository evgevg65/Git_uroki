
Основные команды
git init
Позволяет проинициализировать репозиторий в текущей папке
git status
Показывает текущий статус
git add
Отслеживает изменения файлов
git add index.html — добавляет index.html
git add . — добавляет все файлы
git commit
Сохраняет изменения в коммит
git commit -m 'commit message' — создает коммит с сообщением
git branch
Работа с ветками в репозитории
git branch — показывает список веток
git branch branch-name — создает новую ветку branch-name
git branch -D branch-name — удаляет ветку branch-name
git checkout
Переключается на другую ветку
git checkout branch-name — переключается на последний коммит в ветке
branch-name
git checkout -b branch-name — создает и переключается на ветку branch-name
git merge
Совмещает текущую ветку с выбранной
git merge branch-name — совмещает текущую ветку с branch-name
git config
Конфигурация и параметры git
git config --global user.name — Показывает имя пользователя
git config --global user.name 'new user' — Изменяет имя пользователя
git config --global user.email — Показывает email пользователя
git config --global user.email 'test@mail.ru' — Изменяет email пользователя
git push
Заливает текущие локальные коммиты в удаленный репозиторий
git pull
Забирает изменения с удаленного репозитория в локальный
git clone
Клонирует проект с удаленного репозитория







1. Копируете себе локально репозиторий.
git clone url [folder]
cd [folder]
2. Создаете ветку [branch_name]
git checkout -b [branch_name]
3. Делайте изменения
4. Делаете один коммит или несколько.
git add . (если создавали новые файлы)
git commit -am "Added beautiful fixes"
5. Создаете удаленную ветку.
git push --set-upstream origin [branch_name]
В дальнейшем когда удаленная ветка создана то просто
git push
6. Создаете в интерфейсе github pull-request. Из вашей ветки в master.
8536993405.png
7. автор или вы сами вливаете(merge)/отклоняете(decline) pull-request.
На этом этапе можно добавить комментарии или замечания к коду, что-то исправить.
8.Когда пул-реквест влили, вы локально переключаетесь в master и забираете все изменения
git checkout master
git pull
